\sectioncentered*{Введение}
\addcontentsline{toc}{section}{Введение}

С ростом сложности программного обеспечения с огромной скоростью растёт количество потенциальных ошибок и стоимость разработки. Сфера разработки всегда старается найти наиболее оптимальные способы разрабатывать программные продукты, придумывая новые методологии, архитектуры и подходы. В 1966 году два программиста, Ole Johan и Kristen Nygaard заметили, что в языке программирования \textit{ALGOL} фрейм стека может быть перенесён из стека в кучу, позволяя локальным переменным функции переживать время работы самой функции. Таким образом, функция превратилась в конструктор для класса, а переменные функции стали методами. Дальнейшие эксперименты неизбежно открыли полиморфизм, а позже появились и языки с поддержкой парадигмы \gls{oop}.