\subsection{Инкапсуляция}
\label{sec:incapsulation}

Инкапсуляция позволяет объединить набор данных и методов в одну область видимости, за пределами которого эти методы и данные недосягаемы. Данный подход помогает поддерживать консистентность внутреннего состояния, скрывать сложность системы за небольшим интерфейсом. Во многих языках программирования данный концепт реализуется при помощи модификаторов доступа.

Однако, эта идея не пришла в программирование с появлением \gls{oop}. Язык C имеет прекрасную инкапсуляцию, которая достигается разделением файлов исходного кода на два вида: заголовочные файлы и файлы имлементации. Для примера, рассмотрим простую программу в листингах \ref{incapsulation:fake:1} и \ref{incapsulation:fake:2}.

\begin{code}
	\lstinputlisting{inc/src/point.h}
   \caption{point.h}
   \label{incapsulation:fake:1}
\end{code}

\begin{code}
	\lstinputlisting{inc/src/point.c}
   \caption{point.c}
   \label{incapsulation:fake:2}
\end{code}