\section{Определение ООП}\label{sec:history}

Часто \gls{oop} понимают как <<парадигму, при которой программа моделирует реальный мир>>. В лучшем случае, это определение уклончиво. Тем не менее, это определение показывает одну из черт \gls{oop} -- программный код, спроектированный в терминах ОО, более простой для понимания, так как он имеет более тесную связь с реальным миром.

Другой подход: вывести определение через три основных принципа \gls{oop}:
\begin{itemize}
	\item инкапсуляция;
	\item наследование;
	\item полиморфизм.
\end{itemize}

Предполагается, что \gls{oop} -- правильная комбинация перечисленных выше принципов, однако, так же подразумевается, что именно \gls{oop} ответственен за их появление, что будет оспорено в следующих разделах.

Перед введением всего понятия \gls{oop}, стоит ввести ключевые понятия и рассмотреть основные принципы.