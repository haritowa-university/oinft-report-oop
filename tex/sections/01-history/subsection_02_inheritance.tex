\subsection{Наследование}
\label{sec:inheritance}

По своей сути, \textbf{наследование} -- просто повторное обявление группы переменных и функций в пределах нового скоупа(например, типа). Программисты на C делали подобное действие задолго до ОО языков. В листингах \ref{inheritance:fake:1}, \ref{inheritance:fake:2} и \ref{inheritance:fake:3} отражено дополнение к листингам \ref{incapsulation:fake:1} и \ref{incapsulation:fake:2}, которое объявляет новую структуру \lstinline[language=C]{struct NamedStruct}, <<наследующую>> поля \lstinline[language=C]{double x,y;} и добавляющее новое. Далее эта структура успешно используется в методе \lstinline[language=C]{distance()}.

Подобные трюки были распространены среди программистов до появления ОО. На самом деле, именно так устроено наследование(не множественное) в С++. Другими словами, наследование тоже не появилось вместе с \gls{oop}, однако ОО подход формализовал этот процесс, значительно упростил и сделал приятным в использовании: до поддержки наследования на уровне языка, приходилось дублировать код, преобразовывать типы при передаче их в качестве параметра, постоянно следить за корректностью данных самостоятельно.