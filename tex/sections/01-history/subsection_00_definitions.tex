\subsection{Основные определения}

\textbf{Класс} -- универсальный, комплексный тип данных, состоящий из тематически единого набора «полей» (переменных более элементарных типов) и «методов» (функций для работы с этими полями), то есть он является моделью информационной сущности с внутренним и внешним интерфейсами для оперирования своим содержимым (значениями полей). В частности, в классах широко используются специальные блоки из одного или чаще двух спаренных методов, отвечающих за элементарные операции с определённым полем (интерфейс присваивания и считывания значения), которые имитируют непосредственный доступ к полю. Эти блоки называются «свойствами» и почти совпадают по конкретному имени со своим полем (например, имя поля может начинаться со строчной, а имя свойства — с заглавной буквы). Другим проявлением интерфейсной природы класса является то, что при копировании соответствующей переменной через присваивание копируется только интерфейс, но не сами данные, то есть класс — ссылочный тип данных. Переменная-объект, относящаяся к заданному классом типу, называется экземпляром этого класса. При этом в некоторых исполняющих системах класс также может представляться некоторым объектом при выполнении программы посредством динамической идентификации типа данных. Обычно классы разрабатывают таким образом, чтобы обеспечить отвечающие природе объекта и решаемой задаче целостность данных объекта, а также удобный и простой интерфейс. В свою очередь, целостность предметной области объектов и их интерфейсов, а также удобство их проектирования, обеспечивается наследованием\cite{wiki:oop}.

\textbf{Протокол, интерфейс} -- программная/синтаксическая структура, определяющая отношение между объектами, которые разделяют определённое поведенческое множество и не связаны никак иначе. При проектировании классов, разработка интерфейса тождественна разработке спецификации (множества методов, которые каждый класс, использующий интерфейс, должен реализовывать)\cite{wiki:protocol}.

\textbf{Объект} -- некоторая сущность в компьютерном пространстве, обладающая определённым состоянием и поведением, имеющая заданные значения свойств (атрибутов) и операций над ними (методов). Как правило, при рассмотрении объектов выделяется то, что объекты принадлежат одному или нескольким классам, которые определяют поведение (являются моделью) объекта. Термины <<экземпляр класса>> и <<объект>> взаимозаменяемы\cite{wiki:object}.